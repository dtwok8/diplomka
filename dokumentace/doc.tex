%%%%%%%%%%%%%%%%%%%%%%%%%%%%%%%%%%%%%%%%%%%%%%%%%%%%%%%%%%
%
% Vzor pro sazbu kvalifikační práce
%
% Západočeská univerzita v Plzni
% Fakulta aplikovaných věd
% Katedra informatiky a výpočetní techniky
%
% Petr Lobaz, lobaz@kiv.zcu.cz, 2016/03/14
%
%%%%%%%%%%%%%%%%%%%%%%%%%%%%%%%%%%%%%%%%%%%%%%%%%%%%%%%%%%

% Možné jazyky práce: czech, english
% Možné typy práce: BP (bakalářská), DP (diplomová)
\documentclass[czech,DP]{thesiskiv}

% Definujte údaje pro vstupní strany
%
% Jméno a příjmení; kvůli textu prohlášení určete, 
% zda jde o mužské, nebo ženské jméno.
\author{Kateřina Kratochvílová}
\declarationfemale

%alternativa: 
%\declarationfemale

% Název práce
\title{Nástroj pro automatickou identifikaci KIR alel}

\thanktext{Ráda bych poděkovala Ing. Lucii Houdové, Ph.D. za cenné rady, věcné připomínky, trpělivost a ochotu, kterou mi v průběhu zpracování této práce věnovala.}
% 
% Texty abstraktů (anglicky, česky)
%
\abstracttexten{The text of the abstract (in English). It contains the English translation of the thesis title and a short description of the thesis.}

\abstracttextcz{Text abstraktu (česky). Obsahuje krátkou anotaci (cca 10 řádek) v češtině. Budete ji potřebovat i při vyplňování údajů o bakalářské práci ve STAGu. Český i anglický abstrakt by měly být na stejné stránce a měly by si obsahem co možná nejvíce odpovídat (samozřejmě není možný doslovný překlad!).
}

% Na titulní stranu a do textu prohlášení se automaticky vkládá 
% aktuální rok, resp. datum. Můžete je změnit:
%\titlepageyear{2016}
%\declarationdate{1. března 2016}

% Ve zvláštních případech je možné ovlivnit i ostatní texty:
%
%\university{Západočeská univerzita v Plzni}
%\faculty{Fakulta aplikovaných věd}
%\department{Katedra informatiky a výpočetní techniky}
%\subject{Bakalářská práce}
%\titlepagetown{Plzeň}
%\declarationtown{Plzni}

%%%%%%%%%%%%%%%%%%%%%%%%%%%%%%%%%%%%%%%%%%%%%%%%%%%%%%%%%%
%
% DODATEČNÉ BALÍČKY PRO SAZBU
% Jejich užívání či neužívání záleží na libovůli autora 
% práce
%
%%%%%%%%%%%%%%%%%%%%%%%%%%%%%%%%%%%%%%%%%%%%%%%%%%%%%%%%%%

% Zařadit literaturu do obsahu
\usepackage[nottoc,notlot,notlof]{tocbibind}

% Umožňuje vkládání obrázků
\usepackage[pdftex]{graphicx}

% Odkazy v PDF jsou aktivní; navíc se automaticky vkládá
% balíček 'url', který umožňuje např. dělení slov
% uvnitř URL
\usepackage[pdftex]{hyperref}
\hypersetup{colorlinks=true,
  unicode=true,
  linkcolor=black,
  citecolor=black,
  urlcolor=black,
  bookmarksopen=true}

% Při používání citačního stylu csplainnatkiv
% (odvozen z csplainnat, http://repo.or.cz/w/csplainnat.git)
% lze snadno modifikovat vzhled citací v textu
\usepackage[numbers,sort&compress]{natbib}

%%%%%%%%%%%%%%%%%%%%%%%%%%%%%%%%%%%%%%%%%%%%%%%%%%%%%%%%%%
%
% VLASTNÍ TEXT PRÁCE
%
%%%%%%%%%%%%%%%%%%%%%%%%%%%%%%%%%%%%%%%%%%%%%%%%%%%%%%%%%%
\begin{document}
%
\maketitle
\tableofcontents

\chapter{Úvod}
\chapter{Geny}
V každé buňce lidského organismu, konkrétně v buněčném jádře, je možné nálest 46 chromozomů. Jeden chromozom představuje stočenou dlouhou molekulu DNA (Deoxyribonuklenovou kyselinu). Všech 46 chromozomů obsahuje okolo 100 000 genů. Drobný segment DNA, který řídí buněčnou funkci je právě gen. Konkrétní forma genu je alela. \citep{en_smith}
\section{Nomenaklura}
akorat jeste pred to by teda chtělo hodit jak vubec vypada genom

\section{Imunitní systém}
Imunitní systém chrání organismus před škodlivinami. Skládá se ze dvou hlavních částí vrozené imunity a získané imunity. Jiné oznčení pro vrozenou imunitu může být přirozená, neadaptivní nebo antigenně nespecifická. Jiné označení pro získanou imunitu je specifická nebo adaptivní.
Pro tuto práci je důležitý fakt že NK buňky patří do přirozené imunity. NK buňky budou rozebírány dále v textu.

Vrozená imunita veškeré informace jsou neměnně zapsány v DNA 
- odpovídá po každíém setkání s antigenem stejným mechanismy nemá paměť
- buňky se nechází neustále v kry a takže je aktivace v případě potřeby takřka okamžitá (minuty až hodiny)

Specifická imunita
- v genomu jedince obsaženy pouze její základy
- v průběhu vývoje dochází ke změnám genomu jednotlivých buněk, které se pak odraží na jejich fenotypu
- specifická imunita se fyziologicky rozvíjí až po narození
- nefunguje samostatně vždy spolupracuje s přirozenou imunitou

aktivace až po setkání se svým antigenem
pomalejší nástup než nespecifické mechanismy
jiný průběh u opakovaného setkání
schopnost pamatovat si
zdroj wikiskripta

Antigen jsou látky které imunitní systém rozpozná a reguje na ně. Antigen znamená cizorodá částice. Nejčastější antigeny jsou cizorodé látky z vnějšího prostředí.
Antigeny z organismu samého nazýváme endoantigeny (endogenní antigeny). Alergen je exoantigen, který je u vnímavého jedince schopen vyvolat patologickou (alergickou) imunitní reakci.

Antigen prezentující buňky (APC) a MHC systém
APC jsou buňky vlastního těla schopné fagocytovat (makrofágy, dendritické buňky, B-lymfocyty) – co pozřou, to naštípou na krátké peptidické sekvence a vystaví na svém povrchu k „posouzení“
kromě těchto „vzorků“ mají na povrchu i MHC molekuly (z angl. major histocompatibility complex)
MHC jsou vysoce polymorfní a zcela specifické a unikátní pro každého jedince
MHC určují individuální identitu všech tkání – proto můžou působit komplikace spojené např. s odvržením štěpu po transplantaci
největší koncentrace MHC je v leukocytech, proto se u člověka používá spíše zkratka HLA (z angl. human leukocyte antigens)
více o MHC najdete například na Wikipedii
teprve komplex MHC molekuly s antigenem vystavený na povrchu buňky aktivuje příslušný T-lymfocyt
\section{Imunitní systém, HLA a non-HLA geny}

Human leucocyte antigen(HLA) je genetický systém, který je primárně zodpovědný za rozeznávání vlastního od cizorodého. Tento systém je složen právě z jednotlivých HLA genů rozpoznávající antigeny (cizorodé částice). Pokud HLA gen přijde do styku s antigenem je antigen zničen.
HLA obsahuje pravděpodovně i geny odpovědné za intenzitu imunitní odpovědi.

HLA je rozsáhlý komplex genů, které determinují (určují, rozpoznávjí????) povrchové molekuly (antigeny) umístěné v plazmatické membráně buněk

Hlavní fyziologickou funkcí molekul MHC je předkládat antigeny nebo jejich fragmenty buňkám imunitního systému, především T-lymfocytům (prezentace antigenu je prvním předpokladem pro rozvoj imunitní reakce a tím obrany proti napadení mikroorganismy). Pomocí těchto molekul buňky imunitního systému vzájemně kooperují.

Non-HLA geny jsou geny které se nepodílejí na základní funkci HLA systému.
\\základní rozdíl mezi HLA a non-HLA a kir

Non-HLA geny jsou geny které se nepodílejí na základní funkci HLA systému. Z III třídy jsou to všechny, z II žádný a z I je to směs. Zjednodušeně můžeme říci, že geny které nejsou HLA jsou non-HLA. Tyto geny souvisejí též s funkcí imunitního systému, ne však vylučně s funkcí HLA. 



\subsection{Jak vypadá genom} 
Genová oblast HLA komplexu, se nalézá na krátkem raménku 6. chromozomu (6p21.31), zaujímá úsek dlouhý 3600 kb
(3,6cM), tedy přibliţně jednu tisícinu genomu. Obsahuje 224 genů; 128 funkčních genů
a 96 pseudogenů a patří k regionům s nejvyšší genovou hustotou.

Uprostřed HLA oblasti se nachází úsek o velikosti 1 Mb, ve kterém bylo identifikováno na
70 genů, které se funkčně ani strukturně nepodobají HLA molekulám. Navzdory této
skutečnosti se vžilo označení geny III. třídy, přičemž některé geny původně zařazené do
této třídy jsou nověji označovány jako geny IV. třídy (viz. výše).

HLA-6.Chromozom a KIR 19.chromozom
udíž se segregují nezávisle a
HLA shodní dárci s příjemce mají obvykle různé složení KIR genů (Fryčová)


\section{10/10}
Ta je postavena na případě typizace 5 lokusů (HLA-A/B/C/DRB1/ /DQB1). Vstupním parametrem samotného vyhledávání je míra shody (match) či definované neshody (mismatch). Navržená metoda je platná nejen pro úplnou míru shody 10/10 (shoda HLA-A/B/C/DRB1/DQB1), ale i menší, např. 8/8 (HLA-A/B/C/DRB1), či požadovanou neshodu na konkrétních lokusech, např. 9/10 HLA-A mismatch. 

Během vyhledávání se hodnotí shoda obou alel v lokusech HLA-A, HLA-B, HLA-C,
HLA-DRB1 a HLA-DQB1. Cílem je najít dárce, který bude s příjemcem shodný v 10 znacích
z 10. V závislosti na pacientově stavu a nízké pravděpodobnosti najít včas shodného
nepříbuzného dárce je možné tolerovat odchylky v jednom nebo dvou znacích (9/10,
8/10). Každá odchylka však zvyšuje riziko rozvoje potransplantačních komplikací.
\\
dědičnost HLA znaků
\\
Každý člověk má tzv. fenotyp neboli soubor HLA znaků, který je složen právě
ze dvou haplotypů. Každý z haplotypů je tvořen sadou antigenů obsahujících konkrétní
alely. Polovinu těchto znaků zdědíme od matky a polovinu od otce.
Z hlediska transplantace se v současné době považují za nejdůležitější (a proto
se také nejpřesněji vyšetřují) HLA antigeny I. třídy A, B, C a antigeny II. třídy DR a
DQ. Existuje ale řada dalších – tzv. minoritních antigenů, které dosud nejsou
dostatečně probádány, a jejich vliv na průběh transplantace se teprve zkoumá.
V současnosti je požadavek na míru shody 10/10 neboli v pěti HLA antigenech,
konkrétně v (HLA -A, -B, -C, -DRB1 a –DQB1). Nejmenší možná shoda představuje
6/10 v genech (HLA -A, -B, -DRB1), ale zde bohužel pro pacienta vzniká smrtelné
riziko odvržení štěpu.
\\
\\
Počet teoreticky možných kombinací HLA znaků u člověka dosahuje několika
miliard. Je známo, že některé tkáňové typy (kombinace znaků) se vyskytují v určitém
národě či oblasti častěji, jiné jsou extrémně vzácné. Protože se jednotlivé znaky dědí,
shodu mezi dvěma jedinci najdeme nejsnáze v pokrevním příbuzenstvu. Od rodičů na
potomky se příslušná polovina znaků předává obvykle ve zmíněné kompletní sadě
(haplotypu).
Pro zjednodušení je uveden příklad, podle kterého je dle genetických zákonů
možné dědit jednu ze čtyř možných variant výše zmíněných druhů HLA antigenů mezi
sourozenci (obr. 3.2).
\section{Natural Killer}
Při transplantaci se mohou objevit reakce štěpu proti hostuteli nebo relaps onemocnění.(Návrat nemoci)
Podle nedávných studii kromě HLA genů ovlivňují výsledky přijetí i non HLA geny jedním z nich může být i KIR (Kiler immunoglobulit like receptor  tohle je asi blbě napsaný.)
Takže když je pak více 10/10 nebo 9/10 schody tak by se rozhodovalo podle HLA
Jelikož jsou geny kódovány na různých chrmozomech (HLA 6 a KIR 19) takže HLA schodní dárci s příjemcem mají různé složení KIR genů.
V případěch kdy je více schodných HLA dárců tak by se pak porovnávalo KIR.
KIR má dva haplotypy A a B .. a pak asi můžeš dělat kombinace AA, AB a BB.

NK buňky jsou lymfocyty nespecifického imunitního systému.
Nejdůležitější vlastností je schopnost rozlišit zdravé buňky od buněk infikovaných virem či transformovaných v nádorové buňky a efektivně je likvidovat.

Bylo zjištěno, že
specifické složení motivů centromerních a telomerních B haplotypů KIR genů přispívá
k ochraně před relapsem a zvyšuje šanci na úplné vyléčení AML.

Velká buňka imunitního systému, nepotřebuje antigen aby začala zabíjet. 
-nespecifická imunita - vrozená, neadaptivní - veškeré potřebné informace zapsaná v DNA. Odpovídá při každém setkání s antigenem stejně - nemá paměť -> tady si to pročiřečí
KIR jsou receptory na povrchu NK buněk, 
NK zabíjejí na nazákladě interakce mezi kir receptorem a HLA molekulou na povrchu buňky

NK buňky maji shopnost identifikovat buňky vlastního MHC systému (HLA I.třídy) ktere jsou normálně exprimovány prakticky na všech buňkách v těle. 
Nádorové a některé virem napadené buňky potlačují
expresi HLA I. třídy a tím se brání napadení cytotoxickými T lymfocyty.
Molekuly HLA I. třídy rozpoznávají NK buňky pomocí pozitivních a negativních receptorů, které mohou inhibovat nebo naopak aktivovat NK buňky k „zabíjení“

V užším slova smyslu se jako ligand označuje signální molekula, která se váže na vazebné místo cílového proteinu. Ligand, který je schopný po navázání na receptor vyvolat fyziologickou odpověď, se nazývá agonista, ten, který je schopen se vázat, ale odpověď nespouští, je antagonista

Zjednodušeně: NK buňky neustále systematicky zjišťují přítomnost čí absenci příslušný HLA ligand pro své KIR receptory. 
Pokud je HLA molekula přítomna, pak dojde k vazbě KIR-ligand HLA a protože za normální okolností převládají inhibiční KIR nad aktivačními, tak nedojde ke spuštění cytoxické reakce NK buněk. Jestliže receptory KIR nenalzenou příslušný ligand HLA (vlastní molekulu HLA) aktivační KIR receptory převládnou nad inhibičními a je spuštěna náležitá cytoxická kaskáda.
lymfocyty 
bílá krvinka je leukocyt
- typ bbílé krvinyky 
- T a B lymfocyty - specifická imunita
- NK buňky nespecifická imunita
- vznikají v z lymfatických kmenových buňek v kostní dřeni
Aha takže lymfatické řečiště je více propustné proto to co nejde do cév jde sem pak se to odfiltruje a pak se to vrací do krevního řečiště.

KIR jsou na povrchu NK buňek a kde jsou teda NK buňky? 
NK je v podstatě lymfocyt a to je typ bílé krvinky. jo a nebudou teda spíš  v lymfatické uzlině? 
leukocyty 1. granulocyty - neutrofilní, bazofilní a eozinogilní
		2. agranulocyty - lymfocyty a monocyty
		
neutrofilní granulocyty jsou schopny vycestovat z kapilár do místa zánětu
přeměněné monocyty přítomné v játrech v tělních dutinách (hrudní, bříšní), ve slezině vy lymfatických uzlinách a kostní dřeni
\section{Jak funguje HLA}
\section{Jak funguje non-HLA}

\section{Bordel pro prvni kapitolu}
Takže to vypadá že nejdřív se najde shoda HLA a pak se ještě doděláváv KIR shoda.
Proč KIR? pOrotože roste počet důkazů vluvu genů KIR že mají vliv na výsledky transplance  při leukemii
HLA je na 6. chormozomu KIR je 19 chomozomu. tudiž se segregují nezávisle a Hla shodní dárci s příjemcem mají obvykle různé složení KIR genů
Nesmírná variabilita alel tohoto systému ztěžuje úspěšnost allogeních transplantací. 

\textbf{HLA}
jen zkopírováno a je ta i hezkej obrázek z %https://is.cuni.cz/webapps/zzp/detail/49783?lang=en
Genová oblast HLA komplexu, se nalézá na krátkem raménku 6. chromozomu (6p21.31), zaujímá úsek dlouhý 3600 kb
(3,6cM), tedy přibliţně jednu tisícinu genomu. Obsahuje 224 genů; 128 funkčních genů
a 96 pseudogenů a patří k regionům s nejvyšší genovou hustotou.

Uprostřed HLA oblasti se nachází úsek o velikosti 1 Mb, ve kterém bylo identifikováno na
70 genů, které se funkčně ani strukturně nepodobají HLA molekulám. Navzdory této
skutečnosti se vţilo označení geny III. třídy, přičemţ některé geny původně zařazené do
této třídy jsou nověji označovány jako geny IV. třídy (viz. výše).

\textbf{HLA nomenklatura}
HLA nomenklatura - zase jen skopírováno
Vysoký stupeň polymorfizmu HLA systému zohledňují platné zásady pro označování HLA
alel dané Světovou zdravotnickou organizací WHO (WHO nomenklatura). Princip je
jednoduchý: Kaţdá alela je definována písemným označením lokusu následovaným
hvězdičkou (HLA-DRB1*), a poté kombinací 4 číslic (*0401), přičemţ první dvojčíslí
určuje sérologickou specifitu dané alely, druhé pak označuje alelu na základě její
aminokyselinové sekvence. Případné páté číslo charakterizuje tzv. “tichou“ variantu alely,
tzn. záměnu nukleotidů bez změny aminokyselinové sekvence.

\textbf{Dědičnost}
HLA geny jsou děděny autozomálně kodominantně a vykazují mendelistický typ
dědičnosti. Počet rekombinací v HLA systému je řídký, vyskytuje se přibliţně v 1 %
případů a častěji u ţen. Celá oblast od HLA-F aţ po HLA-DP se přenáší z rodičů na
potomstvo jako haplotyp. V rámci rodiny se mohou vyskytnout teoreticky 4 různé
kombinace rodičovských haplotypů, takţe sourozenci mohou být navzájem buď HLA
identičtí, haploidentičtí (mají jeden haplotyp, v druhém se liší), anebo rozdílní. Rodiče jsou
vůči svým dětem vţdy haploidentičtí [5]. Z genetického hlediska významný fenomén
představuje existence vazebné nerovnováhy (linkage disequilibrium) v rámci HLA. Mnoho
HLA genů se nalézá v tak těsné blízkosti, ţe se přenášejí z rodičů na potomky téměř vţdy
společně. V důsledku této skutečnosti se v populaci vyskytují některé kombinace alel
různých genů častěji, neţ by se očekávalo. Vazebná nerovnováha je významným faktorem
v asociaci HLA antigenů s chorobami, protoţe mnohá onemocnění se v jejím důsledku
váží s více antigeny.

Non-HLA geny
Non-HLA geny jsou geny které se nepodílejí na základní funkci HLA systému. Z III třídy jsou to všechny, z II žádný a z I je to směs. Zjednodušeně můžeme říci, že geny které nejsou HLA jsou non-HLA. Tyto geny souvisejí též s funkcí imunitního systému, ne však vylučně s funkcí HLA. 

lymfocyty 
bílá krvinka je leukocyt
- typ bbílé krvinyky 
- T a B lymfocyty - specifická imunita
- NK buňky nespecifická imunita
- vznikají v z lymfatických kmenových buňek v kostní dřeni
Aha takže lymfatické řečiště je více propustné proto to co nejde do cév jde sem pak se to odfiltruje a pak se to vrací do krevního řečiště.

KIR jsou na povrchu NK buňek a kde jsou teda NK buňky? 
NK je v podstatě lymfocyt a to je typ bílé krvinky. jo a nebudou teda spíš  v lymfatické uzlině? 
leukocyty 1. granulocyty - neutrofilní, bazofilní a eozinogilní
		2. agranulocyty - lymfocyty a monocyty
		
neutrofilní granulocyty jsou schopny vycestovat z kapilár do místa zánětu
přeměněné monocyty přítomné v játrech v tělních dutinách (hrudní, bříšní), ve slezině vy lymfatických uzlinách a kostní dřeni

KIR
KIR jsou teda jak na HLA tak na non-HLA? Je to součástí genu
- řadí se do přirozené (nespecifické) imunity narozdíl od B-buněk a T-buněk.
- NK buňky představují 10-15\% lymfocitů v periferní krvi
- jsou to buňky které reagují rychle a efektivně likvidují především nádorové buňky a buňky infokované virem

NK nemají antigenné specifické receptory, jak rozeznávají abnormální buňky? 
NK buňky identifikují molekuly vlastního MHC systému 

 jmenovitě HLA I. třídy, které jsou normálně exprimovány
prakticky na všech buňkách v těle. Nádorové a některé virem napadené buňky potlačují
expresi HLA I. třídy a tím se brání napadení cytotoxickými T lymfocyty (Restifo, 1993).
Snížená exprese HLA I. třídy činí abnormální buňky citlivé k cytotoxicitě NK buněk (Karre,
1986). Molekuly HLA I. třídy rozpoznávají NK buňky pomocí pozitivních a negativních 
receptorů, které mohou inhibovat nebo naopak aktivovat NK buňky k „zabíjení“

Stručně lze shrnout, že NK buňky s potenciálem iniciovat cytotoxickou aloreakci používají
KIR receptory jako inhibiční, směrem k „vlastním“, zdravým buňkám. Pokud však příslušný
vlastní ligand HLA na cílové buňce chybí, pak dochází k iniciaci cytotoxické reakce. Proces
interakce KIR/HLA a mechanismus regulace cytotoxicity NK buněk se jako 

recptory imunoglobinové (protilátka - protein, který je schopen jako součást imunitního systému identifikovat a zneškodnit cizí objekty - bakterie a viry) v těle. Protilátky jsou nositeli humorální imunity. Jsou to krevní bílkoviny vznikající v mízní tkáni.  povahy nacházejících se na povrchu Natural killers buněk a některých T-buněk (Variabilita v sekvenci).

KIR3D - prej tři skupiny ale to fakt divně popsaný (českej článek) něco s imunoglobulinovými doménami
KIR2D

funkce KIR - 

these genes are endcoded on chromosome 19. NK zabíjejí na základně interakce mezi KIR receptorem a HLA molekulou na povrchu buňek. Mohou mít různé podoby.

HLA i KIR jsou na různých chromozomech proto se segregují nezávisle a HLA schodni darci maji obvykle různé složení KIR genů
\\
\\
\textbf{Struktura nukleových kyselin} \\
jen skopírované z %https://www.wikiskripta.eu/w/Struktura_nukleov%C3%BDch_kyselin \\
Nukleové kyseliny (polynukleotidy) jsou tvořeny dlouhými řetězci (mono)nukleotidů, vzájemně spojených fosfodiesterovými vazbami. Řadíme je k tzv.heteropolymérům, neboť jsou sestaveny z různých typů základních jednotek. Tato skutečnost je podstatná pro uchovávání a předávání informace, což je základní funkce nukleových kyselin v organismu. Homopolyméry (např. glykogen) obsahují pouze jeden typ monoméru (v našem případě glukózu), a tak nemohou plnit informační funkci.

\section{Sekvence DNA}
Je posloupnost písmen představující přimární strukturu reálné nebo hypotetické molekuly čí vlákna DNA, které má kapacitu nést informaci.

Používaná písmena A, C, G a T reprezentují čtyři nukleotidy ve vláknu DNA – adenin, cytosin, guanin a thymin, lišící se 
typem báze kovalentně vázané k fosfátové páteři. Posloupnost libovolného množství nukleotidů většího než čtyři lze nazývat 
sekvencí. Obvykle se sekvence vypisuje bez mezer, např. AAAGTCTGAC, ve směru 5 -> 3. Vzhledem k biologickým funkcím, které 
mohou záviset na kontextu, sekvence buďto mají anebo nemají smysl a jsou tedy kódující nebo nekódující DNA. Typem nekódující 
sekvence DNA je také tzv. „junk DNA“.

TO je z wiki bacha na to.


\chapter{Sekvenační metody získávání DNA dat}

Někdy se sekvenují pouze jisté části genomu které mají pro výzkumníka v daném okamžiku význam.

Sekvenování DNA je souhrný termín pro biochemické metody, jímiž se zjišťuje pořadí nukleových bází (A, C, G, T) v sekvencí DNA. Tyto sekvence jsou součástí dědičné informace v jádru.
Adenin s thyminem a cytosins s guaninem.

zjišťvání přímární struktury nukleových kyselin (sekvencování)

Užitečné nejen ve výzkumu ale i v diagnostice nemocí či forenzní medicíně. 

\section{Porovnání vhodného dárce}
V případě nepříbuzenských transplantací se vybírají potenciální dárci, kteří nemají s daným
pacientem žádný děděný haplotyp. Snahou je najít takového dárce, který má shodné, přestože
děděné od jiných rodičů, HLA antigeny. Informace o tom, jak jsou alely haplotypicky
uspořádány obvykle chybí, proto je vždy nutná typizace maximálním rozlišením ve více
HLA lokusech. Zjišťovaný minimální rozsah HLA shody se v jednotlivých transplantačních
centrech liší. V současné době je u nepříbuzného páru požadována typizace vysokým
rozlišením v lokusech HLA – A, B, C, DR a DQ (http://www.efiweb.eu/efi-
committees/standards-committee.html). Pokud pacient a dárce mají stejné alely na všech
těchto lokusech, hovoříme o shodě 10/10. Při jedné neshodě se jedná o shodu 9/10, při dvou o
shodu 8/10.
K typizaci se nejčastěji používá PCR – SSP (PCR se sekvenačně specifickými primery) či
SBT (sequence based typing) technika, v posledních letech se stává zlatým standardem přímá
sekvenace (SBT) HLA genu.
fričová
U HLA - A, B, C, DR a DQ požaduje se typizace v těchto lokusech. Pokud má dárce shodu ve všechn lokusech hovoříme o shodě 10/10 při jedné neshodě je to 9/10.
\\
U Kir jsou 2 hlavní typy haplotyp A a B, které jsou definovány typem a počtem specifických KIR genů. Neexistuje žádné jednoduché univerzální kriterium definující a odlišující tyto haplotypy. 
Sekvenančí metody s elíší především rychlostí a cenou.
\section{Sanger sequencing}
 K sekvenaci se použtívá gelová elektrofézy
 použitelná k sekvenování krátké sekvence jednovláknové DNA. 
 využívá biologického procesu replikace DNA
 Vybraná sekvence se vloží do reakční směsi s radioaktivně označným primer
 
 
 
\section{NGS next-generation sequencing}
Je rychlé a relativné nenáročné zprácování jednotlivých vzorků. Tisíce až miliony sekvencí mohou být produkovány během jednoho sekvenčního procesu. K popularitě této metody nepomohla i komerciaze cenově dostupních stolních sekvenátorů.

\section{Read}
In DNA sequencing, a read is an inferred sequence of base pairs (or base pair probabilities) corresponding to all or part of a single DNA fragment. A typical sequencing experiment involves fragmentation of the genome into millions of molecules, which are size-selected and ligated to adapters. The set of fragments is referred to as a sequencing library, which is sequenced to produce a set of reads. Je to z wiki zase
\\
\\
V DNA sekvenování, read je odvozená sekvece párů bází odpovídající celému fragmentu DNA nebo jeho části
To znamená že read je kus DNA který by mohl odpovídat nějakého konkrétnímu genu? 
\\
\\
Pak tam ještě bylo psaný něco o read lenght
Sekvenační technologie se liší? v délce vyrobenybch readů. 
Ready díkjy 20-40 párů bází (bp) jsou ultrakrátké
Typická sekvenační metoda vytváří ready délky 100 až 500 bp

Sekvenančí platforma (Illumina) - podle toho se pak připravuje ta sekvenační knihovna


DNA knihovny - podle wikiskripta

DNA knihovny jsou kolekce klonovaných DNA fragmentů genomu určitého organismu (cDNA), které jsou skladovány uvnitř hostitelských organismů (zejména bakterií). cDNA (copy DNA, complementary DNA) je získávána přepisem z mRNA pomocí enzymu reverzní transkriptázy.

Kvalita knihovny
Při přípravě sekvenční knihovny je důležité získat co nejvyšší úroveň složitosti. Jinými slovy, je důležité, aby konečná knihovna co nejvíce odrážela jedinečnost výchozího materiálu. Tento výsledek lze získat především omezením počtu segmentových duplikací. Čím kratší jsou fragmenty, tím vyšší je pravděpodobnost, že jsou fragmenty méně specifické a mohou se zarovnat na více než jednom lokusu referenční sekvence. Složitost knihovny lze tedy v podstatě měřit procentem duplicitních čtení, které jsou přítomny v sekvenčních datech

READY - zase wikipedie
In DNA sequencing, a read is an inferred sequence of base pairs (or base pair probabilities) corresponding to all or part of a single DNA fragment. A typical sequencing experiment involves fragmentation of the genome into millions of molecules, which are size-selected and ligated to adapters. The set of fragments is referred to as a sequencing library, which is sequenced to produce a set of reads


\chapter{Analyza dostupných bioinformatických nástrojů pro zpracování NGS dat}

\section{ART}
ART (next-generation sequencing read simulator) je sada simulačních nástrojů, které generují syntetické ready, jako kdyby byli získány sekvenováním pomocí NGS. Nástroj ART dokáže simulovat ready ze sekvenátorů Illuminas, 454 společnosti Roch a SOLid od společnosti Applied biosystém. Ready, vytvořené nástrojem ART jsou používány pro testování a analýzů nástrojů zpracovávající právě NGS sekvence jako například zarovávání (nástroj Bowtie). 
\\
\\
ART je implementován v jazyce C++ a je dostupný s licencí GPL verze~3. Je dostupný pro operační systémy Linux, MacOs a Windows. Je možné ho používat i jako C++ package.
\\
\\
Data získána z FN Plzeň byla sekvenována nástrojem Illuminas proto i syntetické ready budou simulovat tento sekvenátor.    
 Výstupy se čtou ve formátu FASQ a zarovnání ve formátu ALN. může generovat zarovánávání také ve formátu SAM nebo UCS BED. \cite{art}

\subsection{bordel}
ART is freely available to public. The binary packages of ART are available for three major operating systems: Linux, Macintosh, and Windows. ART is also available as Platform-independent C++ source packages. Each package includes programs, documents and usage examples.

ART simuluje ready napodobobáním skutečných procesů sekvenování s empirickým chybovým modelem nebo quality profiles summarized from large recalibrated sequencing data
ART může také simlovat čtené pomocí uživatelského vlastního read arror modelu nebo quality profiles

TODO - tohle úplně nechápu ART podporuje simulaci jedno párových, dvou párovcýh tří hlavních komerčních sekvenčních platfoem 
Výstupy se čtou ve formátu FASQ a zarování ve formátu ALN. 
ART může také generovat zarovnávání ve formátu SAM nebo UCSC BED
ART lze použít společně se simulátory varient genomů VarSim 
\\
to je odtud %https://www.niehs.nih.gov/research/resources/software/biostatistics/art/index.cfm
454 sekvenování je pyrosekvenování, které cycklicky testuje přítomnost každého ze čtyř nukleotidů DNA (T, A, C, G)

SOLid ke kódování 16 různých dinukleotidů používá čtyči fluoresenční barevná barviva, každé barvivo kóduje čtyři dinukleotidy
 


tak jsem stáhla normálně nejnovější verzi z niehs.nih.gov a podle instrukcí co byli v souboru INSTAL dala % ./configure && make && make install


musí se brát v potaz že z toho generátoru nikdy nebudou data taková jako reálná.. realná budou horší 




\section{Bowtie}
Bowtie je rychlý a paměťové efektivní nástroj pro zarovnávání krátkých sekvencí DNA na velké genomy. Indexace pomocí Burrows-Wheelere transformace dovoluje zarovnávání více než 25 milionů readů za CPU hodinu pro lidský genom s pamětí příblížně 1.3 gigabajtů. Bowtie přidáváa k Burrows-Wheeler technice backtracking algoritmus pro sledování nekonzistence. ??


\subsection{Bordel}
Bowtie je napsanej v c++ a používá knihovnu seqAn

Na lidském genomu je nástoj Bowtie v porovnání s nástoji Maq a SOAP rychlejší. 
Citlovost má bowtie srovnatelné s nástojem SOAP a o něco menší než Maq. Ale je možnost pomocí příkazové řádky zvýšit citlivost na úkor rychlosti běhu progamu.
Oproti SOAP bowtie potřebuje méně paměti 1.3 GB RAM. 
Bowtie zarovnává  25 milionů readů za hodinu. může běžet paralelně.

indexi vytváří permanentní a lze je použít napříč běhy 
pro lidský genom je to 2.2 GB takže ho lze distribuovat přes internet
rychlost a malá paměť způsobuje především Burrows wheeler v kombinaci s backtrackingem.

Podrporuje standardní vstupní formáty FASQ a FASTA.

Bowtie je open source.

 
na stránkách elixir-europe což je oragnizace co má dávat dohromady všechny vědecký veci a bla bla.



Tak tam je přímo Bowtie \cite{bowtie}

\subsection{Bowtie 2}
Note that SOAP2 and Bowtie do not permit gapped alignment of unpaired reads.
 memory footprint of Bowtie 2 (3.24 gigabytes)
 Bowtie 2 by mělo být vhodnější pro delší ready než Bowtie1.
 We extracted a random subset of 1 million reads from each and aligned them with BWA-SW and Bowtie 2. We did not align with Bowtie, BWA or SOAP2 because those tools are designed for shorter reads.
Bowtie už je překonanej nejenom Bowtie2 ale i BWA.
Bowtie2 je podle studie znatelně lepší než Bowtie, SOAP2.
tyhle výsledky jsou na syntetických readech

vypadá to že bowtie 2 už nepoužívá tamten index ale používá nějaký Full-text minute index–assisted search což vypadá že je kombinace burrows wheelera a ještě něčeho.
We found that Bowtie 2, a method that combines the advantages of the full-text minute index and SIMD dynamic programming, achieved very fast and memory-efficient gapped alignment of sequencing reads

je zase open source
\cite{bowtie2}


šla jsem přes docker docker image ls - zobrazi vsechny image pak docker run a ID image
sudo docker run -i -t 3c2b9a287f82 /bin/bash
sudo docker ps -a

Tak jsem nakonec žádnej docker nepotřebovala a stáhla jsem to tady %http://bowtie-bio.sourceforge.net/tutorial.shtml
 po kliknuti na bowtie binary release.

na strance 25.4 je řečeno o hledání tch nejlepších zarovnání a je tam možnost --best ale že je dvakrát nebo třikrát pomalejší než normální mod.. a jde o to že najde první přijatelný a to označní kdežto při tom best prohledá co nejvíc a hledá to nejlepší i mezi těma přijatelnýma a to je pomalý.

takže zarovnání by mohlo být teoreticky namapování na referenční gen???
 
% 
% PRO ANGLICKOU SAZBU JE NUTNÉ ZMĚNIT
% CITAČNÍ STYL!
%
\nocite{*}
\bibliographystyle{csplainnatkiv}
{\raggedright\small
\bibliography{literatura}
}

\end{document}
